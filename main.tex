\documentclass{beamer}

\usepackage{amsmath}
\usepackage[english]{babel}
%\usepackage[sorting=none]{biblatex}
\usepackage{hyperref}
\usepackage{unicode-math}
\usepackage{mathtools}
\usepackage{derivative}
%\usepackage{appendixnumberbeamer}

\usetheme[progressbar=frametitle]{metropolis}

\setmainfont{Stix Two Text}
\setmathfont{Stix Two Math}

\input{math}

\title{A theoretical framework for quantum-optical communication - towards CV-QKD}
\date{\today}
\author{Bodo Kaiser}
\institute{Ludwig-Maximilians-Universität München}

\begin{document}
	\maketitle
	
	\begin{frame}
		\begin{center}
			\textbf{Prelude}
			
			What is light?
		\end{center}
	\end{frame}
	
	\begin{frame}
		\begin{center}
			\textbf{Postlude}
			
			How does light interact with matter?
		\end{center}
	\end{frame}
\end{document}